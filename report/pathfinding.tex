\documentclass[12pt, a4paper]{article}

% Preamble

\usepackage[utf8]{inputenc}
\usepackage[english]{babel}
\usepackage{fullpage}
\usepackage[parfill]{parskip}

\usepackage[hidelinks]{hyperref}
\bibliographystyle{ieeetr}
\usepackage{courier}
\usepackage{amsfonts}
\usepackage{amsmath}
\usepackage{mathtools}
\DeclarePairedDelimiter{\floor}{\lfloor}{\rfloor}

\title{Pathfinding, PKD 2014}
\author{Babak Mohebbolhoggeh, Nils Lerin and Robin Eklind}

% Document

\begin{document}

\maketitle

\pagebreak

\tableofcontents

\pagebreak

% === [ Introduction ] =========================================================

\section{Introduction}

Pathfinding is the act of locating a path from point A to point B. A common
special case of pathfinding is concerned with locating the shortest possible
path between two points. Pathfinding has many applications and while some are
obvious, such as map directions, others are less apparent. For instance some
network protocols use pathfinding to locate efficient routes for network
traffic.

% --- [ Purpose ] --------------------------------------------------------------

\subsection{Purpose}

There are many ways to implement pathfinding algorithms, each with its own
benefits and drawbacks. Our goal with this project is to research various
pathfinding algorithms and highlight their strengths and weaknesses.

We implement the Dijkstra algorithm since it's a simple pathfinding algorithm,
which is naive but easy to grasp. For performance reasons we implement the A*
algorithm, which has become the de facto algorithm in game development
\cite{astar1, defacto}.

Another goal with this project is to implement a minimal interface for
performing pathfinding.

% === [ Pathfinding ] ==========================================================

\section{Pathfinding}

% Historic (Dijkstra one of the first, A* somewhere during the end of the 70is).

% Graph theory.

% More examples and use cases.

% Mention many different pathfinding algorithms.

% The effects of edge costs.
%    Shortest path, best path, stealth units.

% === [ Design ] ===============================================================

\section{Design}

% General design description.

% The project have been divided into distinct components, below are the major
% ones:
% * vector2. (datatype and interface of 2d array)
% * grid/tilegrid. (datatype and interface of grid)
% * graph. (datatype and interface of graph)
% * pqueue/pqueue_list/queue. (datatype and interface of pqueue)
% * pathfinding. (datatype and interface of pathfinder)
% * pretty.

\subsection{Design choices}

% Array vs list of lists vs ... Graphs vs 2d-arrays.

% Deliberate choice to perform expensive preprocessing of the graph so that the
% pathfinding itself can be faster.

% But we do not preprocess edge costs.

% Prioriy queues: list (simple), binomial heaps (hopefully fast).

% Number of directions (8 directions).

% ~~~ [ Cutting corners ] ~~~~~~~~~~~~~~~~~~~~~~~~~~~~~~~~~~~~~~~~~~~~~~~~~~~~~~

The pathfinding algorithm can either allow or forbit diagonal steps to cut
corners. Obviously this decision may affect the shortest path. \cite{astar2}

% It's not perfect, but the code was structured so a future change wouldn't
% require an extensive rewrite. Special concideration was made to make it easy
% to extend from (x, y) to (x, y, z).

% === [ Algorithms ] ===========================================================

\section{Algorithms}

% --- [ Dijkstra ] -------------------------------------------------------------

\subsection{Dijkstra}

% How it works.

% Backtracking

% Useful to find the shortest path of multiple targets, or targets at unknown
% coordinates.

% Its major drawback is performace...

% Time complexity.

% --- [ A* ] -------------------------------------------------------------------

\subsection{A*}

% How it works.

The main idea behind the A* algorithm is to use a heuristic function which makes
an educated guess about the distance of a node to the goal. Using this
information it is able prioritize the processing of nodes which are more likely
to bring you closer to the goal. This reduces the number of nodes which have to
be processed in average and therefore decreases the total runtime of the average
case.

% Open list, closed list (colors), F, G, H cost, etc.

As mentioned in section \ref{Pathfinding} most pathfinding algorithms can be
described using graph theory. Each node of a graph represents a walkable cell
and the edge between two nodes connects them and makes it possible to perform a
step from one node to the other. The concept of edge costs captures the time it
takes for an entity to step from one node to another, which may vary depending
on the terain of the edge and other factors.

For each node of the graph the A* algorithm keeps track of three different cost
values, namely \textit{F}, \text{G} and \text{H}. The \textit{G} value of a node
\textit{C} represents the cost of traversing the edges from the start node
\textit{S} to the node \textit{C}.

% Time complexity.

% ~~~ [ Heuristics ] ~~~~~~~~~~~~~~~~~~~~~~~~~~~~~~~~~~~~~~~~~~~~~~~~~~~~~~~~~~~

\subsubsection{Heuristics}

\textit{Heuristics using Manhattan distance}

If the pathfinding algorithm overestimates the distance H it may fail to find
the shortest path. For these cases we call the heuristics inadmissible.
\cite{astar2}

\textit{Diagonal heuristics}

Instead of using pure Manhattan distance for the H cost we could use a hybrid
between the Manhattan distance and ''as the crow flies''.

\textit{Example:} Let 's' and 'g' represent the start and goal locations
respectively. Let the cost of a horizontal or vertical step be 10 and the cost
of a diagonal step be 14 ($ \floor{10 \sqrt{2}} $).

\texttt{+-------+ \\
|.....g.| \\
|.s.....| \\
+-------+}

The cost using Manhattan distance between 's' and 'g' is $ 1*10 + 4*10 = 50 $,
since there is one vertical and four horizontal steps from 's' to 'g'.

Using a hybrid heuristic we could combine a horizontal and a vertical step into
a diagonal step which would cost 14 instead of 10+10=20. Using this method the
heuristic could no longer be inadmissible.

The cost using a hybrid heuristic would in this case be $ 1*14 + 3*10 = 44 $,
since there is one diagonal and three horizontal steps from 's' to 'g'.

\textit{Inadmissible heuristics}

% ~~~ [ Priority queues ] ~~~~~~~~~~~~~~~~~~~~~~~~~~~~~~~~~~~~~~~~~~~~~~~~~~~~~~

\subsubsection{Priority queues}

\begin{enumerate}
\item{List}
\item{Binomial Heap}
\end{enumerate}

\textit{Similarities to Dijkstra}

% The Dijkstra algorithm could be implemented using A* with a heuristic function
% which always returns 0.


% ~~~ [ Multiple shortest paths ] ~~~~~~~~~~~~~~~~~~~~~~~~~~~~~~~~~~~~~~~~~~~~~~

\subsubsection{One out of multiple shortest paths}

When two nodes have the same F score one must be chosen over the other. This
choise will not affect the length of the path, but it may affect the path
itself. Another way to put this is that depending on how ties are treated two
different implementations of the A* algorithm may find different paths of equal
length. \cite{astar2}

% === [ User's Manual ] ========================================================

\section{User's Manual}

% TODO: Use cases, examples...

% ref: demo.sml

% Mention the files and what content they contain. Mention the most important
% functions and their arguments.

% === [ Analysis and discussion ] ==============================================

\section{Analysis and discussion}

% Implementation analysis.

% Test cases, how the we verify that the implementation is correct?

% Is there anything that doesn't work. It is possible to walk thorugh diagonal
% walls.

% Benchmarks. Is A* measurabley more performable than Dijkstra. How does the
% performance of list and binomal heaps differ.

\subsection{Future work}

% Preprocess edge costs.

% Disallow diagonal walks through walls.

% Pathfinding in the third dimension.

% Hot queue.

% Formal verifications.

% Make an interlanguage bridge to make the SML API callable from C. ref: mlton

% Make purely functional (if we don't finish this) and measure the performance
% impact.

\pagebreak

\bibliography{references}

\end{document}
