\documentclass[12pt, a4paper]{article}

% Preamble

\usepackage[utf8]{inputenc}
\usepackage[english]{babel}
\usepackage{fullpage}
\usepackage[parfill]{parskip}

\usepackage[hidelinks]{hyperref}
\bibliographystyle{ieeetr}
\usepackage{courier}
\usepackage{amsfonts}
\usepackage{amsmath}
\usepackage{mathtools}
\DeclarePairedDelimiter{\floor}{\lfloor}{\rfloor}

\title{Pathfinding, PKD 2014}
\author{Babak Mohebbolhoggeh, Nils Lerin and Robin Eklind}

% Document

\begin{document}

\maketitle

\pagebreak

\tableofcontents

\pagebreak

% === [ Introduction ] =========================================================

\section{Introduction}

Pathfinding is the act of locating a path from point A to point B. A common
special case of pathfinding is concerned with locating the shortest possible
path between two points. Pathfinding has many applications and while some are
obvious, such as map directions, others are less apparent. For instance some
network protocols use pathfinding to locate efficient routes for network
traffic.

% --- [ Purpose ] --------------------------------------------------------------

\subsection{Purpose}

There are many ways to implement pathfinding algorithms, each with its own
benefits and drawbacks. Our goal with this project is to research various
pathfinding algorithms, highlight their strengths and weaknesses and implement
the A* algorithm, which has become the de facto algorithm in game development
\cite{astar1, defacto}.

% === [ Pathfinding ] ==========================================================

\section{Pathfinding}

% === [ Design ] ===============================================================

\section{Design}

% === [ Algorithms ] ===========================================================

\section{Algorithms}

% --- [ Dijkstra ] -------------------------------------------------------------

\subsection{Dijkstra}

% --- [ A* ] -------------------------------------------------------------------

\subsection{A*}

% ~~~ [ Inadmissible heuristics ] ~~~~~~~~~~~~~~~~~~~~~~~~~~~~~~~~~~~~~~~~~~~~~~

\subsubsection{Inadmissible heuristics}

\textit{Heuristics using Manhattan distance}

If the pathfinding algorithm overestimates the distance H it may fail to find
the shortest path. For these cases we call the heuristics inadmissible.
\cite{astar2}

\textit{Hybrid heuristics (possible solution)}

Instead of using pure Manhattan distance for the H cost we could use a hybrid
between the Manhattan distance and ''as the crow flies''.

\textit{Example:} Let 's' and 'g' represent the start and goal locations
respectively. Let the cost of a horizontal or vertical step be 10 and the cost
of a diagonal step be 14 ($ \floor{10 \sqrt{2}} $).

\texttt{+-------+ \\
|.....g.| \\
|.s.....| \\
+-------+}

The cost using Manhattan distance between 's' and 'g' is $ 1*10 + 4*10 = 50 $,
since there is one vertical and four horizontal steps from 's' to 'g'.

Using a hybrid heuristic we could combine a horizontal and a vertical step into
a diagonal step which would cost 14 instead of 10+10=20. Using this method the
heuristic could no longer be inadmissible.

The cost using a hybrid heuristic would in this case be $ 1*14 + 3*10 = 44 $,
since there is one diagonal and three horizontal steps from 's' to 'g'.

% ~~~ [ Treatment of ties ] ~~~~~~~~~~~~~~~~~~~~~~~~~~~~~~~~~~~~~~~~~~~~~~~~~~~~

\subsubsection{Treatment of ties}

When two nodes have the same F score one must be chosen over the other. This
choise will not affect the length of the path, but it may affect the path
itself. Another way to put this is that depending on how ties are treated two
different implementations of the A* algorithm may find different paths of equal
length. \cite{astar2}

% ~~~ [ Cutting corners ] ~~~~~~~~~~~~~~~~~~~~~~~~~~~~~~~~~~~~~~~~~~~~~~~~~~~~~~

\subsubsection{Cutting corners}

The pathfinding algorithm can either allow or forbit diagonal steps to cut
corners. Obviously this decision may affect the shortest path. \cite{astar2}

% === [ Design choices ] =======================================================

\section{Design choices}

% === [ Implementation ] =======================================================

\section{Implementation}

% === [ User's Manual ] ========================================================

\section{User's Manual}

% TODO: Use cases, examples...

% === [ Analysis and discussion ] ==============================================

\section{Analysis and discussion}

\pagebreak

\bibliography{references}

\end{document}
